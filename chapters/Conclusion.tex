%!TEX root = ../Thesis.tex
\chapter{Conclusion}

The main objective of this project was to design, implement and test a control system and navigation approach. The main tasks were:
a) understanding what is involved in rocket stability in flight and gimbaled engine TVC
b) building a technology demonstrator prototype simulating the rocket to test the control systems on
c) developing a inertial navigation system based on an inertial measurement unit
d) designing a controller that will use the inertial sensor data to control a system suitable for a gimbaled engine
e) testing controller on a technology demonstrator prototype as a substitute for the rocket.
The requirements were: a) investigating sources of errors in the inertial sensor, particularly the gyro drift in flight; b)deciding on an absolute heading (yaw) sensor to correct the gyro orientation; c) navigational attitude representation through a method avoiding singularities; d) investigating navigation filters fit for lower processing power, Arduino class micro-controllers; e) navigation filters with a average deviation of maximum 1.5 degrees from the true value, for each axis; f) 1 axis control that can be extended to 2 or 3 axes, g) controller with max 30 percent overshoot and 1.5 sec settling time.


The first task was fulfilled in chapter 2, where it was found that a key component of rocket stability in flight is having its center of pressure located below its center of gravity, as well as stabilizing fins (inherent stability). The other component is active guidance, achieved through a controller. 
Second task was building a technology demonstrator was described in chapter 3, where the building process of the drone is detailed. Third task of developing an inertial navigation system has been detailed in chapter 4, mentioning the different sensor errors, calibration routines and fusion filters. The fourth task has been designing a controller suitable for a gimbaled engine m that would use the navigation data - decisions which are reported in chapter 5, Control section. 
The final task of testing the controller on the technology demonstrator is described in chapter 6, with the tests showing that the controller can successfully stabilize the UAV. 
Requirement a) to d) were discussed in chapter 4, discussing the various types of errors specific to each device, choosing magnetometer as absolute heading measurement, filters based on quaternions as a method to avoid singularities and are fit for lower processing power micro-controllers. 
Requirement e) was validated in chapter 6, during testing, where Mahony filter emerged as the better performing filter, with Madgwick a close second - given steady calibrated data. 
The control requirements f) and g) were fulfilled by using a PID controller, which can be extended for all axes and having found sets of gains which stabilize the system with minimal overshoot and returning it to equilibrium in the least possible time. 
Therefore, all tasks and all requirements were met in full. The objective of understanding and developing a workflow necessary for the concept of an inertial navigation based thrust vector control was achieved.
